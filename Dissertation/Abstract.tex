
Inter-state conflicts are a key area of study in international relations, and have been approached with a variety of techniques, from case studies of individual conflicts to formal analysis of abstract models. In particular, there are a variety of theories as to how states make decisions in the face of conflicts -- such as when to threaten force, when to follow through, and when to capitulate to an opponent's demand. Some scholars have argued that states may be viewed as rational decisionmakers, while others hold that the psychological biases affecting individual leaders play a decisive role. Decisionmaking is challenging to study in part because of its complexity: the decisionmakers may not just be individuals but organizations, following internal procedures and reflecting institutional memory; furthermore, the decisions are often believed to be strategic, reflecting an anticipation of other actors' potential responses to each possible decision.

In this dissertation, I will demonstrate that agent-based models (ABMs) provide a powerful tool to address this complexity. Agents in ABMs can be endowed with a variety of internal decisionmaking models, which can operationalize a variety of theories of decisionmaking, from simple heuristics to perfect strategic rationality. The specific decision model agents utilize may be changed without altering the sub-models governing how the agents interact with one another. This allows us to run models of the same overall interactions utilizing different decisionmaking models and observe how the outcomes differ. Furthermore, if these interactions correspond to real-world events, we may directly see how much explanatory or predictive power the outputs of the model variants provide. If one variant's outputs correspond closer to the empirical data, it provides evidence that the variant's underlying theory is more correct.

I implement two agent-based models, extending well-established prior models of international conflict: the International Interaction Game \citep{bdm_1992} and the Expected Utility Model \citep{bdm_2002}. For each, I start with their original agent decisionmaking model, and develop several variants grounded in relevant theories. I then instantiate the models with historic, empirically-derived data and run them forward to generate sets of simulated outcomes, which I compare to empirical data on the relevant time periods. I find that non-rational models of decisionmaking in the International Interaction Game provide similar explanatory power to the purely rational model, and yield rich satisficing behavior absent in the original model. I also find that the Expected Utility Model variant implementing a \citet{schelling_1966}-inspired model of coercion yields richer dynamics and greater explanatory power than the original model. 
